\documentclass[twoside,12pt]{report}	%automatic offset in margins for binding

\usepackage[utf8]{inputenc}		%to use accents (only used for "Clément", in fact)
\usepackage[frenchb,english]{babel}	%to use proper french formatting when needed while keeping english formatting when not
\usepackage{geometry}			%to change margins
\usepackage[super]{nth}			%for numerical ordering
\usepackage[table]{xcolor}			%to put color in tables

%useless stuff as we use a custom title page
\title{Book of Specifications}
\author{Clément Gillard
	\and Aymeric Alixe
	\and Thibault Chamoy
	\and Thibaut Cherel}

\makeatletter
\renewcommand{\@makechapterhead}[1]{%
  \vspace*{0\p@}%
  {
    \interlinepenalty\@M
    \Huge \bfseries #1\par\nobreak
    \vskip 40\p@
  }}

\renewcommand{\@makeschapterhead}[1]{%
  \vspace*{0\p@}%
  {
    \interlinepenalty\@M
    \Huge \bfseries #1\par\nobreak
    \vskip 40\p@
  }}

\newcommand{\cg}{\cellcolor{gray!50}}%shortcut for gray cells
\newcommand{\cc}{\cellcolor{cyan!50}}%shortcut for blue cells
\newcommand{\cp}[1]{\cellcolor{gray!#1} #1\%}%shortcut for grayscale cells
\newcolumntype{a}{>{\columncolor{gray!80}}c}
\makeatother

\begin{document}

	\newgeometry{margin=2cm}

	\begin{titlepage}
		Clément \bsc{Gillard} Aymeric \bsc{Alixe} Thibaut \bsc{Cherel} Thibault \bsc{Chamoy} \hfill \today
		{\centering
			\vfill%\\[10cm]
			\Huge I WANNA DIE
			\\[1cm]
			\small by the \Large Suicide Squad
			\\[1cm]
			\large Book of Specifications
			\vfill
		}
	\end{titlepage}

	\restoregeometry

	\newpage

	\tableofcontents

	\newpage
	\phantom{}
	\newpage
	\setcounter{page}{4}
	\addcontentsline{toc}{chapter}{\large Introduction}
	\chapter*{Introduction}

		Hello to whoever will read this book of specifications which will introduce to you our game project.
		\\\\
		Our team is called the Suicide Squad and our project 'I WANNA DIE'.
		We will present you, in this book of specification, the different aspects of our project which is, trust me, a bit unusual .
		\\\\
		This game is born in the mind of Clément then Aymeric joined him in his quest , modifying a bit the original idea , then being friends with Thibaut and Thibault and them having no group, they decided to join the group. After some discussions, beers, and trepidant murders, we all mixed our ideas and then all agreed on the different aspects of this project.
		\\\\
		I WANNA DIE is a reflexion and platform game. The player will evolve in a rather friendly environment.
		In fact, we would like to create a game which has a different goal than the usual ones. The game consists of a set of levels where the aim is to kill yourself in order to go to the next harmless level. That turnaround in the usual game mechanics will affect all the game and may disturb the player during his gaming experience.
		\\\\
		The aim of this project is to exceed our skills to present, at the end, a result which we can be proud of. Furthermore,
		we will have to improve ourself in coding, and respect some restrictions that we will find again in the professional world.
		\\\\
		Well, this is the end of this introduction page to our project. We wish you a good reading ! :)
		
	\chapter{Individual Presentations}

		\section{Clément}

			I'm Clément Gillard, the group leader. The goal of this project is to make a functional game as a team, in a defined span of time. As a group, it will teach me how to efficiently work on a group project. As an individual member of the group, it will give me the ability to learn about and work on some technology and algorithms I might never have used without having this project to do as a part of my education at EPITA. As a group leader, it will teach me how to manage a team and its teamwork, just has I might have to do during my professional career as an information technology engineer.

		\section{Thibaut}

			I am Thibaut Chérel, and I am in the class E1 at EPITA. At the beginning of the year, I wanted to do a platform game, because I always loved this kind of games, but I had no idea about who should I do it with. Then I started to think about doing it with Thibault. We heard that Clément and Aymeric, who are both our friends, wanted to be together but they were only 2. So Thibault and I decided to join them. When Clément spoke about a suicide game, I was very excited. We made a lot debates to decide precisely what we would like our game to be. We finally decided to make a reflexion platform game, which looks like my first idea, and everyone was happy with this. I think the project will really help me to work in group, because it is not just a group work of a few weeks ; we will work on this project together during 6 months. We will need a good team work if we want to succeed and produce a good game. The project will also teach me a lot of things, which are needed for the game and that I, as of right now, have absolutely no idea how to do. Furthermore, a long project like this one also improve our ability to work in group and individually, because the project will be spread between us, and I will have to work alone on my part. So I am very excited about this project, just by thinking about all the knowledge I will acquire.

		\section{Aymeric}

			Hello world ! I'm Aymeric, 18 years old and already passionate by information technology.
            \\
			I believe that working in a group is a good way to improve yourself. In fact, it trains people to collaborate and make concessions. Furthermore, the aim of this project is to overpass myself to be able present, in the end, a good result of which i can be proud of. I would like too earn some skills too.
			\\
			I am in this group because I share the same idea with the other members. We all wanted to do something a bit different than usual games, while still being a classic platform and reflexion game like Portal is. Furthermore, I wanted to code a game which had a different goal than survive and win (like all platform game), that's the reason why I joined this group.
		 	\\
			We are all different, but we will all bring our qualities in this project and I am sure we will work well together. Clément is a great developer, and he is not afraid of difficulties. Thibaut is reactive and funny. Thibault is serious and motivated.
			\\
			I'm in charge of the level design because I would like to create the environment where the hero evolves. In every video game, the environment shows the atmosphere of the game, and that is something I really want to emphasize on .

		\section{Thibault}

			Shortly before the distribution of the "Project Information" file, I did not really have a project idea, nor a group to do it with. I just talked to Aymeric and Clément, they suggested to me their idea of a "suicide game", and I was really interested by this idea, it looked interesting to code, so they proposed me to join their group and I accepted. Personally this game will ask me concentration, effort and enough organization to finish this project. But I will also need to learn how to work with a group, organize myself with people and allocate tasks.

	\chapter{Synopsis}

		The game is based after the film "Inception" by Christopher Nolan (a spin-off, you would say). Our character knows that, if he is in a dream and can wake up only by dying, he has no proof that the current world he lives in is the real one. His only way to be sure is to die, so he decided to commit suicide since, it is his only way to be really sure that he is not just dreaming. But in his quest for a real world, he would have to show his ability to commit suicide in dreams, which can be very friendly and a little too safe to die, and his will of traveling between worlds .

	\chapter{Structure}

		Our project is a game, and we want this game to be in 3D. It will be a reflexion and a platform game so we need to make different levels. That is why we need designated level designers. In our game, we will implement sounds and cutscenes. We also need to add an AI for a few non-player characters and a network to play online with other players.
		\\
		To make our game accessible for all, we will create a website, in which there will be a way to download the game, a presentation of the project, like the game, the team members and the evolution of the game. It would also need links to the softwares and the elements, like sounds and images, we will use.
		\\
		We want to make a trailer, a short advertisement, which would tempt people to play the game. To show the progression of the project, this trailer will progress with it, thus the viewer will see new part of the game.
		\\
		To realize this project, we will need to work separately on different task. But those tasks must be put together, so we will use a distributed revision control with Git to allow us to share files and parts of the game easily.
		\\
		This project will also have costs because we will use a lot of softwares like Unity, Visual Studio and Audacity, and we will also need some material resources.

	\chapter{Main points of development}

		We all wanted to do a 3D game , because we wish that the player has more possibilities than the classics jumping, going forward and backward.
		\\
		The game will be like all traditional platform games, the player will be able to interact with some object of the environment in 3D, take bonuses and penalties.
		We are not decided yet to give the possibility to attack in order to kill, it will depend of the needed to have a way to defense.%%%%%%%% wut ? tu peux la faire en français ?
		They will be an introduction chapter, to introduce the player to the game mechanics, and then the level of difficulty will increase at each chapter .
		\\
		To simplify our work, we decided to code in C\# which is more "human friendly" than the other language we were proposed, OCaml.
		For the graphics, we can make our own textures with GIMP or Adobe Photoshop but most of the textures will be taken from the Internet.
		We will work with Unity and Visual Studio and the different tools which are available on Framework.NET in order to ensure that there are no conflict between our coding data.%%%%%%%% pareil
		\\
		The code has to prepared for all cases and cope with the various errors. It has to be easily understandable and optimized.

		\def\arraystretch{1.5}

		\section{Task distribution}
			\begin{tabular}{|l|c|c|c|c|}
				\cline{2-5} \multicolumn{1}{c|}{}	 & \cg Clément	 & \cg Aymeric	 & \cg Thibaut	 & \cg Thibault	\\
				\hline Git							 & \cc X		 & 				 & 				 & 					\\
				\hline Graphics						 & 				 & \cc X		 & 				 & \cc X			\\
				\hline H.U.D.						 & 				 & 				 & \cc X		 & 					\\
				\hline A.I.							 & \cc X		 & 				 & 				 & \cc X			\\
				\hline \LaTeX \& redaction			 & \cc X		 & 				 & 				 & \cc X			\\
				\hline Level design					 & 				 & \cc X		 & \cc X		 & 					\\
				\hline Network						 & \cc X		 & 				 & 				 & \cc X			\\
				\hline Story						 & 				 & 				 & \cc X		 & 					\\
				\hline C\# Scripts					 & 				 & \cc X		 & 				 & \cc X			\\
				\hline Website						 & 				 & \cc X		 & 				 & 					\\
				\hline Sounds						 & 				 & \cc X		 & \cc X		 & 					\\
				\hline Trailer \& cutscenes			 & 				 & 				 & \cc X		 &					\\
				\hline
			\end{tabular}

		\section{Objectives for each presentation}
			\begin{tabular}{|l|c|c|c|}
				\cline{2-4} \multicolumn{1}{c|}{}	 & \nth{1} Presentation	 & \nth{2} Presentation	 & \nth{3} Presentation	\\
				\hline Camera						 & \cp{100}			 & \cp{100}				 & \cp{100}				\\
				\hline Languages					 & \cp{0}				 & \cp{0}				 & \cp{100}				\\
				\hline Decor						 & \cp{30}				 & \cp{100}				 & \cp{100}				\\
				\hline AI							 & \cp{0}				 & \cp{30}				 & \cp{100}				\\
				\hline Player Interface				 & \cp{30}				 & \cp{80}				 & \cp{100}				\\
				\hline Character					 & \cp{70}				 & \cp{100}				 & \cp{100}				\\
				\hline Multiplayer					 & \cp{0}				 & \cp{20}				 & \cp{100}				\\
				\hline Sounds						 & \cp{20}				 & \cp{100}				 & \cp{100}				\\
				\hline Levels						 & \cp{20}				 & \cp{60}				 & \cp{100}				\\
				\hline Game Settings				 & \cp{0}				 & \cp{30}				 & \cp{100}				\\
				\hline Website						 & \cp{100}				 & \cp{100}				 & \cp{100}				\\
				\hline Trailer						 & \cp{0}				 & \cp{100}				 & \cp{100}				\\
				\hline
			\end{tabular}

		\section{Material costs}
			\begin{tabular}{|a|lrc|}
				\hline \multicolumn{1}{|c|}{\cellcolor{gray}Software} & \multicolumn{3}{c|}{\cellcolor{gray}Cost}	\\
				\hline Unity				& \$ & 1,500.00 &  (offered by EPITA)	\\
				\hline GIMP					& \multicolumn{3}{c|}{FREE}				\\
				\hline Adobe Suite			& \$ & 12.00 & (monthly)				\\
				\hline Blender				& \multicolumn{3}{c|}{FREE}				\\
				\hline \LaTeX \ softwares\footnotemark[1] & \multicolumn{3}{c|}{FREE}\\
				\hline Sony Vegas Pro		& \$ & 599.95 &							\\
				\hline Sublime Text			& \$ & 70.00 &							\\
				\hline Visual Studio Pro	& \$ & 499.00 & (offered by EPITA)		\\
				\hline Audacity				& \multicolumn{3}{c|}{FREE}				\\
				\hline Notepad++			& \multicolumn{3}{c|}{FREE}				\\
				\hline Mono					& \multicolumn{3}{c|}{FREE}				\\
				\hline \cellcolor{gray}Total\footnotemark[2]	& \cellcolor{gray}\$ & \cellcolor{gray}2,728.95	& \multicolumn{1}{|c}{\cellcolor{gray}(\$1,999 by EPITA)}\\
				\hline Real total\footnotemark[2]\textsuperscript{,}\footnotemark[3] & \$ & 729.95  & \multicolumn{1}{|c}{}\\
				\cline{1-3}
			\end{tabular} 
			\vfill
			\footnotetext[1]{This includes many softwares on many different platforms. All of them are free.}
			\footnotetext[2]{Calculation over 5 months}
			\footnotetext[3]{Softwares already provided by EPITA not included}

	\chapter*{Conclusion}

		We are sure of the success of this project, because we all are very motivated about it. We know our actual skills, and thus we know this project is feasible in six months.
		\\
		We also have a lot of ideas to add in our project, we are focused on the principals parts, but if we finish those parts earlier than expected, we could still add a lot of things.
		\\
		So we hope you will support us in this project, because it became very important to us.
		\\
		We also would like to thank you for reading this book of specifications, and maybe for helping us with it.
		\\\\\\
		\phantom{} \hfill Clément \bsc{Gillard} Aymeric \bsc{Alixe} Thibaut \bsc{Cherel} Thibault \bsc{Chamoy}.
		\\
		\phantom{} \hfill - aka. the \bsc{Suicide Squad}
		
	\addcontentsline{toc}{chapter}{\large Conclusion}

\end{document}
